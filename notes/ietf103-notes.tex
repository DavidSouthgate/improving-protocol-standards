\documentclass[10pt]{article}
\usepackage[a4paper, total={6.5in, 11in}, includefoot, heightrounded]{geometry}
\usepackage{adjustbox}
\usepackage{array}
\usepackage{color}

\title{Structured Representations in IETF Documents:\\Notes from IETF 103}
\date{\today}

\newcommand{\todo}[1]{\textbf{\textcolor{red}{To do -- #1}}}
\newcommand{\question}[1]{\vspace{4mm}\noindent\textbf{Q:} \textit{#1}}

\begin{document}
\maketitle

These notes are from informal discussions and surveys carried out at IETF 103 in Bangkok,
Thailand, in November 2018. 

\section{Adoption of current representation formats}

Survey participants were given a list of the structured representation formats commonly
found in IETF documents\footnote{ABNF, XML, ASN.1, C code, JSON, CBOR, TLS presentation
language, YANG, MIB, state transition diagrams, ASCII packet header diagrams, protocol
sequence (ladder) diagrams}. While the sample size for the survey was too small to draw
conclusions about broad trends, none of the participants indicated that they had seen
CBOR in IETF documents. All of the other formats had been seen by at least one participant,
with the all participants having seen C code, YANG models, and ASCII packet header diagrams.

When asked if these structured representations improved the utility of the documents they
were included in, the broad consensus was that they did -- but not significantly. The
representations were good to have, but they do not replace the body of the document
(i.e., the English prose). Further, there was concern that structured representations
-- especially code and pseudocode -- were actually detrimental to readers understanding of
the document. There were two reasons given for this: (i) the (pseudo)code can be overly
complicated, and more difficult to understand than the English prose version; and (ii)
because one is not typically generated from the other, differences can arise between the
pseudo(code) and the English prose. On the positive side, participants noted that some
representations -- particularly diagrams, such as packet header and sequence diagrams --
were useful for clearly defining the syntax of the protocol, and giving structure and
context to the English prose that accompanied them.

Participants were split on whether or not the current level of adoption of structured
representations ws sufficient. The learning curve involved in understanding these
languages -- or at least having sufficient knowledge to make good use of them in
documents -- limits their adoption. In addition, adding a structured representation
requires repeating yourself: if some protocol feature has already been described in
English, you don't want to encode the same in code.

Other miscellaneous feedback:
\begin{itemize}
	\item A lack of tooling makes adopting structured representations more challenging in
	      a lot of respects: correctness, repetition, lack of grammar for ad-hoc
	      pseudocode languages
	\item Text-based protocols typically use ABNF, there isn't a similarly widely adopted
	      format for binary protocols
	\item Would be good to write one (in English or a structured language), and then
	      generate the other. This would avoid duplication and inconsistency.
\end{itemize}

\section{Generating parsers from standards documents}

\section{Other application domains}

\end{document}