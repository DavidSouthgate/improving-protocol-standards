\documentclass[10pt,twocolumn,a4paper]{article}
\usepackage[l2tabu,orthodox]{nag}
\usepackage[utf8x]{inputenc}
\usepackage[british]{babel}
\usepackage[babel=true]{microtype}
\usepackage{amsmath}
\usepackage[all]{onlyamsmath}
\usepackage{newtxtext} 
\usepackage{newtxmath} 
\usepackage{upquote}
\usepackage{graphicx}
\usepackage{url}
\usepackage[caption=false]{subfig}
\usepackage{booktabs}
\usepackage{bytefield}
\usepackage{listings}
\usepackage{algorithm}
\usepackage{algpseudocode}
\usepackage{color}
\usepackage{fullpage}
\usepackage{no-par-indent}
\usepackage{alltt}
\frenchspacing
\newcommand{\todo}[1]{\textit{\textcolor{red}{[To do: #1]}}}
\newcommand{\code}[1]{\texttt{#1}}
%==================================================================================================
\begin{document}
\title{The Glasgow Packet Language: Intermediate Representation and Execution Model}
\author{
  Stephen McQuistin\\University of Glasgow
\and 
  Colin Perkins\\University of Glasgow
}
\date{\today}
\maketitle
%==================================================================================================
\begin{abstract}

  This memo defines the intermediate representation used in the
  implementation of the Glasgow Packet Language. It also describes
  the execution model used to parse protocols described in that
  language.

\end{abstract}
%==================================================================================================
\section{Introduction}

% Paragraph 1: Motivation. At a high level, what is the problem area you
% are working in and why is it important? It is important to set the larger
% context here. Why is the problem of interest and importance to the larger
% community?



% Paragraph 2: What is the specific problem considered in this paper? This
% paragraph narrows down the topic area of the paper. In the first
% paragraph you have established general context and importance. Here you
% establish specific context and background.

In order to generate a parser for a protocol, it's necessary to describe
the format of protocol data units (PDUs).
These can be described by a set of types that represent the objects to be
parsed, along with constraints on the parsed values.
If there are multiple ways of describing these types, it's necessary to
have a common \emph{intermediate representation} into which all those input
formats can be converted, and from which parsers can be generated.

% Paragraph 3: "In this paper, we show that...". This is the key paragraph
% in the introduction - you summarize, in one paragraph, what are the main
% contributions of your paper, given the context established in paragraphs 
% 1 and 2. What's the general approach taken? Why are the specific results
% significant? The story is not what you did, but rather:
%  - what you show, new ideas, new insights
%  - why interesting, important?
% State your contributions: these drive the entire paper.  Contributions
% should be refutable claims, not vague generic statements.

We describe such an intermediate representation in this memo.
It specifies a set of internal types used by the parsers, and constructors
for representable types that describe the PDUs to be parsed.
The execution model of the parsers is also described.

% Paragraph 4: What are the differences between your work, and what others
% have done? Keep this at a high level, as you can refer to future sections
% where specific details and differences will be given, but it is important
% for the reader to know what is new about this work compared to other work
% in the area.



% Paragraph 5: "We structure the remainder of this paper as follows." Give
% the reader a road-map for the rest of the paper. Try to avoid redundant
% phrasing, "In Section 2, In section 3, ..., In Section 4, ... ", etc.


%==================================================================================================
\section{Terminology}

The key words ``MUST'', ``MUST NOT'', ``REQUIRED'', ``SHALL'', ``SHALL
NOT'', ``SHOULD'', ``SHOULD NOT'', ``RECOMMENDED'', ``NOT RECOMMENDED'',
``MAY'', and ``OPTIONAL'' in this document are to be interpreted as
described in \cite{RFC2119,RFC8174} when, and only when, they appear
in all capitals, as shown here.



The intermediate representation is specified in terms of serialised JSON
\cite{RFC7159} objects that describe the packet formats to be specified.
An interpreter for the intermediate representation will read a sequence of
such objects and perform the operations described, in the order that they
objects are read.

%==================================================================================================
\section{Type System}

Types can be \emph{internal} or \emph{representable}. A representable type
describes something that can be parsed or serialised. An internal type is
an artefact of the runtime, and cannot be parsed or serialised.

Representable types MUST implement the \code{Sized} trait.

Each type has \emph{name}. It implements one or more \emph{traits} that
define \emph{methods} that can operate on instances of that type.

%--------------------------------------------------------------------------------------------------
\subsection{Internal Types}



% - - - - - - - - - - - - - - - - - - - - - - - - - - - - - - - - - - - - - - - - - - - - - - - - -
\subsubsection{Primitive Types}
\label{sec:primitives}

The \code{Nothing} type is the empty type. It implements no traits.

The \code{Boolean} type is a boolean value, either \code{true} or \code{false}.
It implements the traits \code{Value}, \code{Equality} (two \code{Boolean}s are equal if
they have the same value), and \code{BooleanOps}.

The \code{Size} type is a positive integral value, denoting the size of an
instance of a representable type. It implements the traits \code{Value},
\code{Equality}, \code{Ordinal}, and \code{ArithmeticOps}.

The primitive types are built-in to the runtime, and are not explicitly
defined in the JSON form of the intermediate representation.

% - - - - - - - - - - - - - - - - - - - - - - - - - - - - - - - - - - - - - - - - - - - - - - - - -
\subsubsection{Traits}
\label{sec:traits}

Traits define the methods that can operate on instances of types that
implement the trait.

A trait definition includes the name of the trait, and the definitions
of the methods. It defines a new type that has the specified name.

The name of a trait is formed of upper and lower case ASCII letters,
digits, or dollar signs (\code{A-Za-z0-9\$}); it MUST begin with an
upper case letter (\code{A-Z}). Trait names are defined in the global type
namespace.  It is an error to define a trait that has the same name as an
existing type.

Each method definition include the method name, return type, parameters.
Methods take one or more parameters.
The first parameter of a method MUST have the name \code{self} and MUST
have an unspecified type.
Other parameters, if any, MUST have names that are unique within that
method definition, and can have specified or unspecified types.
The return type of a method can be specified or unspecified.  
It is an error to define two methods that have the same name.

Method names and parameter names are formed of upper and lower case ASCII
letters, digits, dollar signs, or underscores (\code{A-Za-z0-9\$\_});
they MUST begin with a lower case letter (\code{a-z}).

When a type implements a trait, any unspecified types are set to the
implementing type.

When a method is called on an instance of a type that implements a trait,
that instance will be passed as the first parameter of the method.

In the JSON form of the intermediate representation, a trait definition is
represented as in the following:
\footnotesize
\begin{alltt}
  {
    "construct"   : "Trait"
    "name"        : "\emph{name}",
    "methods"     : [
      \{
        "name"       : "\emph{method name}",
        "parameters" : [
          \{
            "name" : "\emph{parameter name}",
            "type" : "\emph{parameter type}"
          \},
          ...
        ],
        "return\_type" : "\emph{return type}"
      \},
      ...
    ]
  }
\end{alltt}
\normalsize
Unspecified types are indicated by \code{null} values in the JSON definition
(this is distinct from a return value of \code{Nothing} that needs to be
explicitly stated).

The runtime defines the \code{Value} trait. This defines methods
that can be used to get and set the value of an instance of a type. 
Its definition is equivalent to the following definition in the JSON
form of the intermediate representation:
\footnotesize
\begin{alltt}
  \{
    "construct" : "Trait"
    "name"      : "Value",
    "methods"   : [
      \{
        "name"        : "get",
        "parameters"  : [
          \{"name" : "self",  "type"   : null\}
        ],
        "return\_type" : null
      \},
      \{
        "name"        : "set",
        "parameters"  : [
          \{"name" : "self",  "type"   : null\},
          \{"name" : "value", "type"   : null\}
        ],
        "return\_type" : "Nothing"
      \},
      
    ]
  \}
\end{alltt}
\normalsize

The runtime defines the \code{Sized} trait. This defines a method
that can be used to get the size, in bits, of an instance of a type. 
Its definition is equivalent to the following definition in the JSON
form of the intermediate representation:
\footnotesize
\begin{alltt}
  \{
    "construct" : "Trait"
    "name"      : "Sized",
    "methods"   : [
      \{
        "name"        : "size",
        "parameters"  : [
          \{"name" : "self",  "type"   : null\}
        ],
        "return\_type" : "Size"
      \}
    ]
  \}
\end{alltt}
\normalsize

The runtime defines the \code{IndexCollection} trait. This defines methods that can
be used to get and set the values of elements of an array-like data structure, which has
multiple, numerically indexed elements. Its definition is equivalent to the following
definition in the JSON form of the intermediate representation:
\footnotesize
\begin{alltt}
  \{
    "construct" : "Trait"
    "name"      : "IndexCollection",
    "methods"   : [
      \{
        "name"        : "get",
        "parameters"  : [
          \{"name" : "self",  "type"   : null\},
          \{"name" : "index", "type"   : "Size"\}
        ],
        "return\_type" : null
      \},
      \{
        "name"        : "set",
        "parameters"  : [
          \{"name" : "self",  "type"   : null\},
          \{"name" : "index", "type"   : "Size"\}
          \{"name" : "value", "type"   : null\}
        ],
        "return\_type" : "Nothing"
      \},
      \{
        "name"        : "length",
        "parameters"  : [
          \{"name" : "self",  "type"   : null\}
        ],
        "return\_type" : "Size"
      \}
    ]
  \}
\end{alltt}
\normalsize

The runtime defines the \code{Equality} trait. This defines methods that
can be used to compare to instances of a type for equality.
Its definition is equivalent to the following definition in the JSON
form of the intermediate representation:
\footnotesize
\begin{alltt}
  \{
    "construct" : "Trait"
    "name"      : "Equality",
    "methods"   : [
      \{
        "name"        : "eq",
        "parameters"  : [
          \{"name" : "self",  "type"   : null\},
          \{"name" : "other", "type"   : null\}
        ],
        "return\_type" : "Boolean"
      \},
      \{
        "name"        : "ne",
        "parameters"  : [
          \{"name" : "self",  "type"   : null\},
          \{"name" : "other", "type"   : null\}
        ],
        "return\_type" : "Boolean"
      \}
    ]
  \}
\end{alltt}
\normalsize

The runtime defines the \code{Ordinal} trait. This defines methods that
can be used to compare the values of instances of a type. 
Its definition is equivalent to the following definition in the JSON
form of the intermediate representation:
\footnotesize
\begin{alltt}
  \{
    "construct" : "Trait"
    "name"      : "Ordinal",
    "methods"   : [
      \{
        "name"        : "lt",
        "parameters"  : [
          \{"name" : "self",  "type"   : null\},
          \{"name" : "other", "type"   : null\}
        ],
        "return\_type" : "Boolean"
      \},
      \{
        "name"        : "le",
        "parameters"  : [
          \{"name" : "self",  "type"   : null\},
          \{"name" : "other", "type"   : null\}
        ],
        "return\_type" : "Boolean"
      \},
      \{
        "name"        : "gt",
        "parameters"  : [
          \{"name" : "self",  "type"   : null\},
          \{"name" : "other", "type"   : null\}
        ],
        "return\_type" : "Boolean"
      \},
      \{
        "name"        : "ge",
        "parameters"  : [
          \{"name" : "self",  "type"   : null\},
          \{"name" : "other", "type"   : null\}
        ],
        "return\_type" : "Boolean"
      \}
    ]
  \}
\end{alltt}
\normalsize

The runtime defines the \code{BooleanOps} trait. This defines methods
that can be used to perform boolean operations on, and between, instances
of a type.
Its definition is equivalent to the following definition in the JSON
form of the intermediate representation:
\footnotesize
\begin{alltt}
  \{
    "construct" : "Trait"
    "name"      : "BooleanOps",
    "methods"   : [
      \{
        "name"        : "and",
        "parameters"  : [
          \{"name" : "self",  "type"   : null\},
          \{"name" : "other", "type"   : null\}
        ],
        "return\_type" : "Boolean"
      \},
      \{
        "name"        : "or",
        "parameters"  : [
          \{"name" : "self",  "type"   : null\},
          \{"name" : "other", "type"   : null\}
        ],
        "return\_type" : "Boolean"
      \},
      \{
        "name"        : "not",
        "parameters"  : [
          \{"name" : "self",  "type"   : null\}
        ],
        "return\_type" : "Boolean"
      \}
    ]
  \}
\end{alltt}
\normalsize

The runtime defines the \code{ArithmeticOps} trait. This defines methods
that can be used to perform arithmetic operations on instances of a type.
Its definition is equivalent to the following definition in the JSON
form of the intermediate representation:
\footnotesize
\begin{alltt}
  \{
    "construct" : "Trait"
    "name"      : "ArithmeticOps",
    "methods"   : [
      \{
        "name"        : "plus",
        "parameters"  : [
          \{"name" : "self",  "type"   : null\},
          \{"name" : "other", "type"   : null\}
        ],
        "return\_type" : null
      \},
      \{
        "name"        : "minus",
        "parameters"  : [
          \{"name" : "self",  "type"   : null\},
          \{"name" : "other", "type"   : null\}
        ],
        "return\_type" : null
      \},
      \{
        "name"        : "multiply",
        "parameters"  : [
          \{"name" : "self",  "type"   : null\},
          \{"name" : "other", "type"   : null\}
        ],
        "return\_type" : null
      \},
      \{
        "name"        : "divide",
        "parameters"  : [
          \{"name" : "self",  "type"   : null\},
          \{"name" : "other", "type"   : null\}
        ],
        "return\_type" : null
      \}
    ]
  \}
\end{alltt}
\normalsize

%--------------------------------------------------------------------------------------------------
\subsection{Representable Types}
\label{sec:representable}

The runtime does not define any representable types. Rather, it defines
a number of \emph{type constructors} that allow appropriate types to be
created as needed to represent a protocol.

% - - - - - - - - - - - - - - - - - - - - - - - - - - - - - - - - - - - - - - - - - - - - - - - - -
\subsubsection{Bit String Types}
\label{sec:bit-string}

The bit string type constructor allows construction of types representing
multi-bit values that can be parsed or serialised.

Instances of bit string types implement the \code{Sized} trait. The size
MAY be unspecified.

Instances of bit string types implement the \code{Value} and
\code{Equality} traits. The former allows the value of the \code{BitString}
instantiation to be retrieved and set. The latter allows two instantiations
of \code{BitString} to be compared for equality; they are equal if they
have the same name, value, and \code{size}.

In the JSON form of the intermediate representation, an invocation of the
bit string type constructor is represented as in the following:
\footnotesize
\begin{alltt}
  \{
    "construct" : "BitString"
    "name"      : "\emph{type name}",
    "size"      : \emph{number of bits}
  \}
\end{alltt}
\normalsize
The \code{name} field specifies the name of the newly defined type. The
name is formed of upper- and lower-case ASCII letters, digits, and dollar
signs (\code{A-Za-z0-9\$}).  It MUST begin with an upper case letter
(\code{A-Z}). Type names are defined in the global type namespace. 
It is an error to define the same type more than once.

The \code{size} field MAY be \code{null} in the JSON, to indicate that the
size is unspecified.

% - - - - - - - - - - - - - - - - - - - - - - - - - - - - - - - - - - - - - - - - - - - - - - - - -
\subsubsection{Arrays}

The array type constructor allows creation of types that represent a
sequence of elements of some other type that can be parsed or serialised.
The array type constructor is parameterised by the name of the new array
type, type of the elements it contains, and the length of the new array.
The length MAY be unspecified. 

Arrays implement the \code{Equality} trait. Two arrays are equal if they
have the same element type, \code{length}, \code{size}, and their elements are equal.

Arrays implement the \code{Sized} trait. The size of the array is \code{length}
multiplied by the size of \code{element\_type}.

Arrays implement the \code{IndexCollection} trait. Elements are numbered from $0$ to
$\code{length}-1$. The \code{length} of an array is the number of elements that it
contains.

In the JSON form of the intermediate representation, an invocation of the
array type constructor is represented as in the following:
\footnotesize
\begin{alltt}
  \{
    "construct"    : "Array",
    "name"         : "\emph{type name}",
    "element\_type" : "\emph{element type}"
    "length"       : \emph{number of elements}
  \}
\end{alltt}
\normalsize
The \code{name} field specifies the name of the newly defined type. The
name is formed of upper- and lower-case ASCII letters, digits, and dollar
signs (\code{A-Za-z0-9\$}).  It MUST begin with an upper case letter
(\code{A-Z}). Type names are defined in the global type namespace.
It is an error to define the same type more than once.

The \code{element\_type} field specifies the type of the array elements.
The element type MUST have been previously defined, and can be a BitString,
array, structure type, or enumerated type, or a type derived from such a
type. The \code{element\_type} MUST be specified.
An array cannot contain elements of its own type.

The \code{length} field MAY be \code{null} in the JSON, to indicate that
the length is unspecified.

% - - - - - - - - - - - - - - - - - - - - - - - - - - - - - - - - - - - - - - - - - - - - - - - - -
\subsubsection{Structure Types}

The structure type constructor allows creation of types that represent a
sequence of fields, with possibly different types, that can be parsed and
serialised. The structure type constructor is parameterised by the name of
the structure, details of the fields, and any constraints on the field
values.

In the JSON form of the intermediate representation, invocation of the type
constructor for a structure type is represented as in the following:
\footnotesize
\begin{alltt}
  \{
    "construct"   : "Struct"
    "name"        : "\emph{type name}",
    "fields"      : [
      \{
        "name"       : "\emph{field name}",
        "type"       : "\emph{field type}",
        "transform"  : \emph{transform expression}
      \},
      ...
    ],
    "constraints" : [
      ...
    ],
    "actions" : [
      ...
    ]
  \}
\end{alltt}
\normalsize
The \code{name} field specifies the name of the newly defined type. The
name is formed of upper- and lower-case ASCII letters, digits, and dollar
signs (\code{A-Za-z0-9\$}).  It MUST begin with an upper case letter
(\code{A-Z}). Type names are defined in the global type namespace.
It is an error to define the same type more than once.

The \code{fields} define the contents of the new structure type. Each
field has a name, a type, an expression that indicates if the field is
present in a particular instantiation of the structure type, and a
transform function. Structure types MUST contain at least one field.

Each field has a name that is formed of upper and lower case ASCII letters,
digits, dollar signs, or underscores (\code{A-Za-z0-9\$\_}). The field name
MUST begin with a lower case letter (\code{a-z}).

Each field has a type. The field type MUST have been previously defined.
Fields can be of \code{BitString}, array, structure type, or enumerated type,
or a type derived from such a type. The names of each field MUST be unique
within a structure definition, but several fields can have the same type.

Each field has a \code{transform} member.
The \code{transform} member MAY be \code{null}, indicating that no
transform is performed.
Alternatively, it is represented as a JSON object as follows:
\footnotesize
\begin{alltt}
  \{
    "into\_name" : "\emph{transformed name}",
    "into\_type" : "\emph{transformed type}",
    "using"      : \emph{FunctionInvocation}
  \}
\end{alltt}
\normalsize
If the transform member is specified, it indicates that the original
field is parsed, then the \code{using} function is invoked, then the field
is replaced by a new field called \code{into\_name} with type \code{into\_type}.
The return type of the \code{using} function MUST match \code{into\_type},
and \code{into\_type} MUST have the same size as the original \code{type}
of the field.

A structure type implements the \code{Sized} trait. The size of a structure is the
sum of the sizes of its fields.

A structure type is parameterised by a set of constraints (i.e., boolean
expressions that MUST evaluate to \code{True}) on the
fields. The set of constraints MAY be empty. Expressions are described in
Section \ref{sec:expressions}.

Finally, on the successful parsing of the structure type, the expressions
in \code{actions} are evaluated. The set of actions MAY be empty. The
expressions in the set of actions MUST be \code{Update} tree expressions,
and each expression MUST return \code{Nothing}. Actions are typically
expected to update the parsing context (see Section \ref{sec:context}).

% - - - - - - - - - - - - - - - - - - - - - - - - - - - - - - - - - - - - - - - - - - - - - - - - -
\subsubsection{Enumerated Types}

The enumerated type constructor allows creation of types that represent
data that can exist as one of several possible variants when parsed or
serialised. The type constructor is parameterised by the name of the new
enumerated type, and the types of the variants.

In the JSON form of the intermediate representation, invocation of the type
constructor for an enumerated type is represented as in the following:
\footnotesize
\begin{alltt}
  \{
    "construct"   : "Enum"
    "name"        : "\emph{type name}",
    "variants"    : [
      \{"type" : "\emph{type name}"\},
      ...
    ]
  \}
\end{alltt}
\normalsize
The \code{name} field specifies the name of the newly defined type. The
name is formed of upper- and lower-case ASCII letters, digits, and dollar
signs (\code{A-Za-z0-9\$}).  It MUST begin with an upper case letter
(\code{A-Z}). Type names are defined in the global type namespace.
It is an error to define the same type more than once.

The \code{variants} field is an array that specifies the possible types
that variants of the enumerated type can take. These types MUST have been
previously defined, and the variants array MUST NOT be empty. Variants of
an enumerated type have their own types, but are otherwise unnamed.
Variants can be of BitString, array, structure type, or enumerated type, or
a type derived from such a type.

An enumerated type implements the \code{Sized} trait. The size of an enumerated type
is equal to the size of the instantiated variant. Variants are not
necessarily all the same size.

% - - - - - - - - - - - - - - - - - - - - - - - - - - - - - - - - - - - - - - - - - - - - - - - - -
\subsubsection{Derived Types}

The derived type constructor allows creation of types as extensions of
existing types. A type derived in this way has the same representation
and properties as the type it is derived from, but has a new name and is
distinct from the original type. It MAY also implement additional traits.

The type constructor is parameterised by the name of the new derived type,
the base type from which it is derived, and a (possibly empty) list of any
additional traits the new type implements.

In the JSON form of the intermediate representation, invocation of the
type constructor for a derived type is represented as in the following:
\footnotesize
\begin{alltt}
  \{
    "construct"     : "NewType",
    "name"          : "\emph{type name}",
    "derived\_from"  : "\emph{type name}",
    "implements"    : [
      \{"trait" : "\emph{trait name}"\},
      ...
    ]
  \}
\end{alltt}
\normalsize
The \code{name} field specifies the name of the newly defined type. The
name is formed of upper- and lower-case ASCII letters, digits, and dollar
signs (\code{A-Za-z0-9\$}).  It MUST begin with an upper case letter
(\code{A-Z}). Type names are defined in the global type namespace.

The \code{derived\_from} field indicates the type from which the new type is
derived.  A new type can be derived from an existing \code{BitString},
array, structure type, or enumerated type. It is an error to define a new
type that has the same name as an existing type. A new type cannot be
derived from itself.

The \code{implements} field is an array that specifies any additional
traits that the new type implements compared to the base type. It is an
error to specify traits that are implemented by the base type.
The \code{implements} fields MAY be an empty array if the derived type
does not implement any additional traits compared to the base type.

A common use for derived types is to extend Bit String types with additional
traits, allowing them to represent integral values that can be compared or
have arithmetic operations performed on them. For example, to define a 16-bit
integer type:
\footnotesize
\begin{alltt}
  \{
    "construct"     : "BitString",
    "name"          : "Bits16",
    "width"         : 16
  \},
  \{
    "construct"     : "NewType",
    "name"          : "Int16",
    "derived\_from"  : "Bits16",
    "implements"    : [
      \{"trait" : "Ordinal"\},
      \{"trait" : "ArithmeticOps"\}
    ]
  \}
\end{alltt}
\normalsize

As derived types inherit the traits of their base type, the size of a derived type is
equal to the size of the type it was derived from.

% - - - - - - - - - - - - - - - - - - - - - - - - - - - - - - - - - - - - - - - - - - - - - - - - -
\subsubsection{Functions}
\label{sec:functions}

The function type constructor allows the definition of new function types,
giving the signature of a function to be provided by the implementation of
the protocol. The code comprising the body of the function is not specified. 

A function type has a name and a return type. It takes a number of
parameters, each of which has its own name and type. 
The types of parameters, and the return type, MUST be specified.
The return type \code{Nothing} is used for functions that return no value.

Function names and parameter names are formed of upper and lower case ASCII
letters, digits, dollar signs, or underscores (\code{A-Za-z0-9\$\_}), and
MUST begin with a lower case letter (\code{a-z}).
Function names are defined in the global type namespace.
Each parameter within a function definition MUST have a unique name.

In the JSON form of the intermediate representation, invocation of the
function type constructor is represented as in the following:
\footnotesize
\begin{alltt}
  \{
    "construct"    : "Function"
    "name"         : "\emph{function name}",
    "parameters"   : [
      \{
        "name" : "\emph{parameter name}",
        "type" : "\emph{parameter type}"
      \},
      ...
    ],
    "return\_type"  : "\emph{return type name}",
  \}
\end{alltt}
\normalsize

A function type implements the \code{Sized} trait. The size of the function is that of the
\code{return\_type}.

% - - - - - - - - - - - - - - - - - - - - - - - - - - - - - - - - - - - - - - - - - - - - - - - - -
\subsubsection{Protocols}

The protocol type constructor allows definition of new protocol types, that
describe the types and protocol data units (PDUs) that form a protocol. The
type constructor is parameterised by the name of the protocol, a set of
type definitions, and the set of PDUs used in the protocol.

In the JSON form of the intermediate representation, invocation of the
protocol type constructor is represented as in the following:
\footnotesize
\begin{alltt}
  \{
    "construct"   : "Protocol",
    "name"        : "\emph{protocol type name}",
    "definitions" : [
                      ...
                    ],
    "pdus"        : [
      \{"type" : "\emph{PDU type name}"\},
      ...
    ]
  \}
\end{alltt}
\normalsize
A protocol is the top-level object in the JSON form of the intermediate
representation.

The \code{name} field specifies the name of the newly defined type. The
name is formed of upper- and lower-case ASCII letters, digits, and dollar
signs (\code{A-Za-z0-9\$}).  It MUST begin with an upper case letter
(\code{A-Z}). Type names are defined in the global type namespace.

The \code{definitions} field that contains an array of objects representing
the types and functions used by the protocol. Any representable type, as
defined in Section \ref{sec:representable}, can be included.

The \code{pdus} field contains an array giving the type of the PDUs used
in the protocol. The PDUs array MAY be empty, although this represents a
protocol that is useless. PDUs can be any structure type or enumerated
type. 

%--------------------------------------------------------------------------------------------------
\subsection{The Parsing Context}
\label{sec:context}

The \code{Context} type constructor allows creation of a type that
represents a sequence of fields, possibly of different types, that can be
accessed in the parsing of other types, where needed. The context type
constructor is parameterised by the details of the fields it includes.

The \code{Context} holds internal state for the parser, and does not
represent a PDU, or any part of a PDU, of the protocol being parsed.
The \code{Context} is defined along with the other \code{definitions}
in a \code{Protocol}, however, since it is specific to a protocol.

In the JSON form of the intermediate representation, invocation of the type
constructor for a context type is represented as in the following:
\footnotesize
\begin{alltt}
  \{
    "construct"   : "Context"
    "fields"      : [
      \{
        "name"    : "\emph{field name}",
        "type"    : "\emph{field type}"
      \},
      ...
    ]
  \}
\end{alltt}
\normalsize
At most one context can be instantiated.

Each field has a name that is formed of upper and lower case ASCII letters,
digits, dollar signs, or underscores (\code{A-Za-z0-9\$\_}). The field name
MUST begin with a lower case letter (\code{a-z}).

Each field has a type. That type MUST have been previously defined, and can
be a \code{BitString}, array, structure type, or enumerated type. The names
of each field MUST be unique within a structure definition, but several
fields can have the same type.

%--------------------------------------------------------------------------------------------------
\subsection{Expressions}
\label{sec:expressions}

Expressions are split into two classes: 
tree (\code{MethodInvocation}, \code{FunctionInvocation}, \code{FieldAccess}, and \code{IfElse})
and 
leaf (\code{This}, \code{Context}, and \code{Constant}).

% - - - - - - - - - - - - - - - - - - - - - - - - - - - - - - - - - - - - - - - - - - - - - - - - -
\subsubsection{Tree Expressions}

A \code{MethodInvocation} tree expression is represented in the JSON form of the
intermediate representation as:
\footnotesize
\begin{alltt}
  \{
    "expression"   : "MethodInvocation",
    "target"       : \emph{expression},
    "method"       : "\emph{method name}",
    "arguments"    : [
      \{
        "name"  : "\emph{parameter name}",
        "value" : \emph{expression}
      \},
      ...
    ]
  \}
\end{alltt}
\normalsize
The \code{MethodInvocation} expression evaluates to the result of applying \code{method}
to \code{target}, with the arguments specified in \code{arguments}. The \code{target} is
either a \code{This} or \code{Context} leaf expression, or a tree expression.

A \code{FunctionInvocation} tree expression is represented in the JSON form of the
intermediate representation as:
\footnotesize
\begin{alltt}
  \{
    "expression"   : "FunctionInvocation",
    "name"         : \emph{function name},
    "arguments"    : [
      \{
        "name"  : "\emph{parameter name}",
        "value" : \emph{expression}
      \},
      ...
    ]
  \}
\end{alltt}
\normalsize
The \code{name} MUST refer to a previously defined \code{Function}. The set
of \code{arguments} MUST contain values for all of the parameters specified
in the \code{Function} definition, and the \code{value} expression for each
MUST evaluate to a value matching the type of the parameter as specified in
the \code{Function} definition. The \code{FunctionInvocation} expression
itself evaluates to the \code{return\_type} specified in the
\code{Function} definition.

A \code{FieldAccess} tree expression is represented in the JSON form of the
intermediate representation as:
\footnotesize
\begin{alltt}
  \{
    "expression"  : "FieldAccess",
    "target"      : \emph{expression},
    "field"       : \emph{field name},
  \}
\end{alltt}
\normalsize
When the target is a structure type, returns a reference to the specified
field of the structure.
When invoked on the parsing context object, returns a reference to the
specified field in the parsing context.
It is an error to invoke \code{FieldAccess} on a target that is not a
structure type or the parsing context.

A \code{IfElse} tree expression is represented in the JSON form of the
intermediate representation as:
\footnotesize
\begin{alltt}
  \{
    "expression"  : "IfElse",
    "condition"   : \emph{expression},
    "if\_true"     : \emph{expression},
    "if\_false"    : \emph{expression}
  \}
\end{alltt}
\normalsize
The \code{IfElse} includes a \code{condition} that evaluates to the \code{Boolean} type.
The \code{IfElse} expression evaluates to the \code{if\_true} member, if \code{condition}
is \code{True}. Otherwise, the expression evaluates to the \code{if\_false} member.

% - - - - - - - - - - - - - - - - - - - - - - - - - - - - - - - - - - - - - - - - - - - - - - - - -
\subsubsection{Leaf Expressions}

A \code{This} leaf expression is represented in the JSON form of the
intermediate representation as:
\footnotesize
\begin{alltt}
  \{
    "expression"   : "This"
  \}
\end{alltt}
\normalsize
It evaluates to a reference to the current object (i.e., the object to
which the expression is attached).

A \code{Context} leaf expression is represented in the JSON form of the
intermediate representation as:
\footnotesize
\begin{alltt}
  \{
    "expression"   : "Context"
  \}
\end{alltt}
\normalsize
It expression evaluates to a reference to the parsing context object.

A \code{Constant} leaf expression is represented in the JSON form of the
intermediate representation as:
\footnotesize
\begin{alltt}
  \{
    "expression"   : "Constant",
    "type"         : "\emph{type name}",
    "value"        : \emph{value}
  \}
\end{alltt}
\normalsize
The \code{Constant} expression evaluates to the specified \code{value},
which has type specified by the \code{type} field.

% - - - - - - - - - - - - - - - - - - - - - - - - - - - - - - - - - - - - - - - - - - - - - - - - -
\subsubsection{Example}

An expression comparing to the \code{version} field of the current
structure object to the constant value 2 would be written:
\footnotesize
\begin{alltt}
  \{
     "expression" : "MethodInvocation",
     "target"     : \{
       "expression" : "FieldAccess",
       "target" : \{
         "expression" : "This"
       \},
       "field"    : "version"
     \},
     "method"     : "eq",
     "arguments"  : [
       \{
         "name"  : "other",
         "value" : \{
           "expression" : "Constant",
           "type"       : "Int16",
           "value"      : "2"
         \}
       \}
     ]
  \}
\end{alltt}
\normalsize

%==================================================================================================
\section{Acknowledgements}

This work was supported by the Engineering and Physical Sciences Research
Council (grant EP/R04144X/1).

%==================================================================================================
\bibliographystyle{abbrvurl}
\bibliography{ir}
%==================================================================================================
% The following information gets written into the PDF file information:
\ifpdf
  \pdfinfo{
    /Title        (The Glasgow Packet Language: Intermediate Representation and Execution Model)
    /Author       (Stephen McQuistin and Colin Perkins)
    /Subject      (The Glasgow Packet Language)
    /Keywords     (Parsing, Network Protocols, Packet Formats)
    /CreationDate (D:20180727162600Z)
    /ModDate      (D:20180727162600Z)
    /Creator      (LaTeX)
    /Producer     (pdfTeX)
  }
  % Suppress unnecessary metadata, to ensure the PDF generated by pdflatex is
  % identical each time it is built:
  \ifdefined\pdftrailerid
    % The \pdftrailerid and \pdfsuppressptexinfo macros were both introduced 
    % in pdfTeX 3.14159265-2.6-1.40.17. If one is present, the other will be.
    \pdftrailerid{}
    \pdfsuppressptexinfo=15
  \fi
\fi
%==================================================================================================
\end{document}
% vim: set ts=2 sw=2 tw=75 et ai:
