\documentclass[10pt,twocolumn,a4paper]{article}
\usepackage[l2tabu,orthodox]{nag}
\usepackage[utf8x]{inputenc}
\usepackage[british]{babel}
\usepackage[babel=true]{microtype}
\usepackage{amsmath}
\usepackage[all]{onlyamsmath}
\usepackage{newtxtext} 
\usepackage{newtxmath} 
\usepackage{upquote}
\usepackage{graphicx}
\usepackage{url}
\usepackage[caption=false]{subfig}
\usepackage{booktabs}
\usepackage{bytefield}
\usepackage{listings}
\usepackage{algorithm}
\usepackage{algpseudocode}
\usepackage{color}
\usepackage{fullpage}
\usepackage{no-par-indent}
\frenchspacing
\newcommand{\todo}[1]{\textit{\textcolor{red}{[To do: #1]}}}
%==================================================================================================
\begin{document}
\title{The Glasgow Packet Language: Intermediate Representation and Execution Model}
\author{
  Stephen McQuistin\\University of Glasgow
\and 
  Colin Perkins\\University of Glasgow
}
\date{\today}
\maketitle
%==================================================================================================
\begin{abstract}

% Four sentences:
%  - State the problem
%  - Say why it's an interesting problem
%  - Say what your solution achieves
%  - Say what follows from your solution



\end{abstract}
%==================================================================================================
\section{Introduction}

% Paragraph 1: Motivation. At a high level, what is the problem area you
% are working in and why is it important? It is important to set the larger
% context here. Why is the problem of interest and importance to the larger
% community?



% Paragraph 2: What is the specific problem considered in this paper? This
% paragraph narrows down the topic area of the paper. In the first
% paragraph you have established general context and importance. Here you
% establish specific context and background.



% Paragraph 3: "In this paper, we show that...". This is the key paragraph
% in the introduction - you summarize, in one paragraph, what are the main
% contributions of your paper, given the context established in paragraphs 
% 1 and 2. What's the general approach taken? Why are the specific results
% significant? The story is not what you did, but rather:
%  - what you show, new ideas, new insights
%  - why interesting, important?
% State your contributions: these drive the entire paper.  Contributions
% should be refutable claims, not vague generic statements.



% Paragraph 4: What are the differences between your work, and what others
% have done? Keep this at a high level, as you can refer to future sections
% where specific details and differences will be given, but it is important
% for the reader to know what is new about this work compared to other work
% in the area.



% Paragraph 5: "We structure the remainder of this paper as follows." Give
% the reader a road-map for the rest of the paper. Try to avoid redundant
% phrasing, "In Section 2, In section 3, ..., In Section 4, ... ", etc.


%==================================================================================================
\section{Intermediate Representation}

The key words ``MUST'', ``MUST NOT'', ``REQUIRED'', ``SHALL'', ``SHALL
NOT'', ``SHOULD'', ``SHOULD NOT'', ``RECOMMENDED'', ``NOT RECOMMENDED'',
``MAY'', and ``OPTIONAL'' in this document are to be interpreted as
described in \cite{RFC2119,RFC8174} when, and only when, they appear
in all capitals, as shown here.



The intermediate representation is specified in terms of serialised JSON
\cite{RFC7159} objects that describe the packet formats to be specified.
An interpreter for the intermediate representation will read a sequence of
such objects and perform the operations described, in the order that they
objects are read.

Each JSON object in the intermediate representation MUST contain a member
named \texttt{irobject} that describes the contents of the object.
Whether other members are included in the object depends on the value of
the \texttt{irobject} member.

%--------------------------------------------------------------------------------------------------
\subsection{Protocols}

The top-level object in the intermediate representation is a
\texttt{Protocol}. A \texttt{Protocol} is represented as a JSON object
containing an \texttt{irobject} member with the value \texttt{protocol}.

\begin{verbatim}
  {
    "irobject"    : "protocol",
    "name"        : "RTPv2",
    "definitions" : [
                      ...
                    ],
    "pdus"        : [
       {"type" : "RtpPacket"},
       {"type" : "RtcpPacket"}
    ]
  }
\end{verbatim}

The \texttt{Protocol} object MUST contain a \texttt{name} member.

The \texttt{Protocol} object MUST contain a \texttt{definitions} member
that contains an array of objects representing the types and functions
used by the protocol, as defined in Section \ref{sec:types} and Section
\ref{sec:funcs} of this memo. This array MUST NOT be empty. The definitions
are presented as a single list, since types and functions can reference
each other.

The \texttt{Protocol} object MUST contain a \texttt{pdus} member. This
will be an array listing the types of the protocol data units. Each PDU
MUST be either a structure type or an enumerated type. The array of PDUs
MUST NOT be empty, and their types MUST be declared in the \texttt{types}
array.

%--------------------------------------------------------------------------------------------------
\subsection{Type Definitions}
\label{sec:types}

As the intermediate representation is loaded, it defines other types. Types
MUST NOT be used before they have been defined.
Types can be either aliases, compound types (arrays or structure types), or
enumerated types.

Type names MUST begin with an upper-case letter ('A'...'Z') and can contain
upper- and lower-case letters, numbers, and \verb|$| signs.

% - - - - - - - - - - - - - - - - - - - - - - - - - - - - - - - - - - - - - - - - - - - - - - - - -
\subsubsection{Bits}

A single primitive type is implicitly defined, the \texttt{Bit}.
A Bit may have value 0 or 1, or have an unspecified value. 

% - - - - - - - - - - - - - - - - - - - - - - - - - - - - - - - - - - - - - - - - - - - - - - - - -
\subsubsection{Aliases}

If an intermediate representation JSON object contains an \texttt{irobject}
member with the value \texttt{newtype}, then that object defines a new type
as an alias for an existing type. For example, a type SpinBit could be
defined based on the Bit type as follows:

\begin{verbatim}
  {
    "irobject"    : "newtype",
    "name"        : "SpinBit",
    "derivedFrom" : "Bit"
  }
\end{verbatim}

A type derived in this way has the same representation and properties as
the type it is derived from, but has a new name and is distinct from the
original type.  In the example above, a SpinBit and a Bit are different
types, and cannot be used interchangeably, even though they have the same
representation.

A new type can be derived from any existing type. It is an error to define
a new type that has the same name as an existing type. A new type cannot be
derived from itself.

% - - - - - - - - - - - - - - - - - - - - - - - - - - - - - - - - - - - - - - - - - - - - - - - - -
\subsubsection{Arrays}

If an intermediate representation JSON object contains an \texttt{irobject}
member with the value \texttt{array}, then that object defines a new array
type. For example, a type \texttt{SeqNum}, being an array of 16 elements of
type Bit, could be defined as follows:

\begin{verbatim}
  {
    "irobject"    : "array",
    "name"        : "SeqNum",
    "elementType" : "Bit"
    "length"      : 16
  }
\end{verbatim}

In addition to the \texttt{irobject} member, the object MUST also include
the following members:
\begin{itemize}
  \item \texttt{name} indicates the name of the type being defined. It is
    an error to define a name multiple times.
  \item \texttt{elementType} indicates the type of the elements of the
    array. The \texttt{elementType} MUST NOT be the same as the array's
    type (i.e., the \texttt{name}).
  \item \texttt{length} indicates the number of elements in the array. This
    MAY be set to \texttt{null} to indicate an array of unspecified length.
\end{itemize}

% - - - - - - - - - - - - - - - - - - - - - - - - - - - - - - - - - - - - - - - - - - - - - - - - -
\subsubsection{Structure Types}

If an intermediate representation JSON object contains an \texttt{irobject}
member with the value \texttt{struct}, then that object defines a new
structure type. For example:

\begin{verbatim}
  {
    "irobject"    : "struct"
    "name"        : "<string>",
    "fields"      : [
      {
        "name"      : "...",
        "type"      : "...",
        "isPresent" : <boolean constraint>
      },
      ...
    ],
    "constraints" : [
      ...
    ]
  }
\end{verbatim}

A structure type represents data comprising an ordered sequence of elements
of possibly different types. In addition to the \texttt{irobject} member, 
the object MUST also include the following members:
\begin{itemize}
  \item \texttt{name} indicates the name of the type being defined. It is
    an error to define a name multiple times.
  \item \texttt{fields} is a JSON array, each value of which is an object
    containing three members, \texttt{name}, \texttt{type}, and \texttt{isPresent}.
    The first two indicate the field's name and type. The type MUST have
    been previously defined; the names of each element within the fields
    array MUST be unique, but multiple elements can have the same type.
    The \texttt{isPresent} field accounts for OPTIONAL fields, and gives
    the constraint that specifies whether the field is present in this
    instance of the struct.
    The fields array MUST NOT be empty.
  \item \texttt{constraints} is a JSON array, containing objects that MUST have a
    \texttt{constraint} member whose value is one of \texttt{constant},
    \texttt{field\_name}, \texttt{binary}, or \texttt{ternary}, and that have the
    corresponding structure as described in Section \ref{sec:constraints}. The
    \texttt{constraints} array MAY be empty.
\end{itemize}

% - - - - - - - - - - - - - - - - - - - - - - - - - - - - - - - - - - - - - - - - - - - - - - - - -
\subsubsection{Enumerated Types}

If an intermediate representation JSON object contains an \texttt{irobject}
member with the value \texttt{enum}, then that object defines a new enumerated
type. For example:

\begin{verbatim}
  {
    "irobject"    : "enum"
    "name"        : "<string>",
    "variants"    : [
      {"type" : "<typename>"},
      ...
    ]
  }
\end{verbatim}

An enumerated type represents data that can be one of several possible
variants. In addition to the \texttt{irobject} member, the object MUST
also include the following members:
\begin{itemize}
  \item \texttt{name} indicates the name of the type being defined. It is
    an error to define a name multiple times.
  \item \texttt{variants} is a JSON array, each value of which is an object
    containing a member named \texttt{type} that has a value indicating one
    possible type that they enumerated type can take. The type MUST have
    been previously defined.  The variants array MUST NOT be empty. Each
    variant has a type, but is otherwise unnamed.
\end{itemize}

%--------------------------------------------------------------------------------------------------
\subsection{Function Prototypes}
\label{sec:funcs}

If an intermediate representation JSON object contains an \texttt{irobject}
member with the value \texttt{function}, then that object defines a function
prototype. For example:

\begin{verbatim}
  {
    "irobject"    : "function"
    "name"        : "decrypt",
    "parameters"  : [
      {
        "name" : "enc_payload",
        "type" : "cryptobits"
      },
      {
        "name" : "pn",
        "type" : "full_packet_num"
      }
    ],
    "returnType"  : "bit",
  }
\end{verbatim}

Function names MUST begin with an lower-case letter ('a'...'z') and can contain
upper- and lower-case letters, numbers, and \verb|$| signs.

%--------------------------------------------------------------------------------------------------
\subsection{Constraints}
\label{sec:constraints}

% - - - - - - - - - - - - - - - - - - - - - - - - - - - - - - - - - - - - - - - - - - - - - - - - -
\subsubsection{\texttt{constant}}

Where \texttt{constraint} has the value \texttt{constant}, the JSON object MUST have the
following structure:

\begin{verbatim}
  {
    "constraint"    : "constant",
    "value"         : "<integer>"
  }
\end{verbatim}

All members MUST be set.

% - - - - - - - - - - - - - - - - - - - - - - - - - - - - - - - - - - - - - - - - - - - - - - - - -
\subsubsection{\texttt{field\_name}}

Where \texttt{constraint} has the value \texttt{field\_name}, the JSON object must have
the following structure:

\begin{verbatim}
  {
    "constraint" : "field_name",
    "value"      : "<string>",
    "property"   : "<'value'
                    |'length'
                    |'is_present'>"
  }
\end{verbatim}

\texttt{value} MUST correspond to a field name defined within the \texttt{fields} JSON
array of the enclosing structure object. \texttt{property} refers to one of the three
inherent properties of fields: their value, length, or presence in the parsed structure.

% - - - - - - - - - - - - - - - - - - - - - - - - - - - - - - - - - - - - - - - - - - - - - - - - -
\subsubsection{\texttt{binary}}

Where \texttt{constraint} has the value \texttt{binary}, the JSON object must have
the following structure:

\begin{verbatim}
  {
    "constraint" : "binary",
    "value"      : "<'|' |'&&'|'|' |'^' |'&' |'=='
                    |'!='|'<='|'>='|'<' |'>' |'<<'
                    |'>>'|'+' |'-' |'*' |'/' |'%'>",
    "left"       : "<constraint>",
    "right"	     : "<constraint"
  }
\end{verbatim}

Both \texttt{left} and \texttt{right} MUST be set to constraint objects.

% - - - - - - - - - - - - - - - - - - - - - - - - - - - - - - - - - - - - - - - - - - - - - - - - -
\subsubsection{\texttt{ternary}}

Where \texttt{constraint} has the value \texttt{ternary}, the JSON object's structure is:

\begin{verbatim}
  {
    "constraint" : "ternary",
    "value"      : "?:",
    "cond"       : "<constraint>",
    "true"       : "<constraint>",
    "else"       : "<constraint>"
  }
\end{verbatim}

\texttt{cond}, \texttt{true}, and \texttt{else} MUST be set to constraint objects.

%==================================================================================================
\section{Acknowledgements}

This work was supported by the Engineering and Physical Sciences Research
Council (grant EP/R04144X/1).

%==================================================================================================
\bibliographystyle{abbrvurl}
\bibliography{ir.bib}
%==================================================================================================
% The following information gets written into the PDF file information:
\ifpdf
  \pdfinfo{
    /Title        (The Glasgow Packet Language: Intermediate Representation and Execution Model)
    /Author       (Stephen McQuistin and Colin Perkins)
    /Subject      (The Glasgow Packet Language)
    /Keywords     (Parsing, Network Protocols, Packet Formats)
    /CreationDate (D:20180727162600Z)
    /ModDate      (D:20180727162600Z)
    /Creator      (LaTeX)
    /Producer     (pdfTeX)
  }
  % Suppress unnecessary metadata, to ensure the PDF generated by pdflatex is
  % identical each time it is built:
  \ifdefined\pdftrailerid
    % The \pdftrailerid and \pdfsuppressptexinfo macros were both introduced 
    % in pdfTeX 3.14159265-2.6-1.40.17. If one is present, the other will be.
    \pdftrailerid{}
    \pdfsuppressptexinfo=15
  \fi
\fi
%==================================================================================================
\end{document}
% vim: set ts=2 sw=2 tw=75 et ai:
