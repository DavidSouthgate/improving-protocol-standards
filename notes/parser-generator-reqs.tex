\documentclass[10pt,twocolumn,a4paper]{article}
\usepackage[l2tabu,orthodox]{nag}
\usepackage[utf8x]{inputenc}
\usepackage[british]{babel}
\usepackage[babel=true]{microtype}
\usepackage{amsmath}
\usepackage[all]{onlyamsmath}
\usepackage{newtxtext} 
\usepackage{newtxmath} 
\usepackage{upquote}
\usepackage{graphicx}
\usepackage{url}
\usepackage[caption=false]{subfig}
\usepackage{booktabs}
\usepackage{bytefield}
\usepackage{listings}
\usepackage{algorithm}
\usepackage{algpseudocode}
\usepackage{color}
\usepackage{fullpage}
\usepackage{no-par-indent}
\usepackage{alltt}
\frenchspacing
\newcommand{\todo}[1]{\textit{\textcolor{red}{[To do: #1]}}}
\newcommand{\code}[1]{\texttt{#1}}
%==================================================================================================
\begin{document}
\title{The Network Representation Language: Parser Generator Requirements}
\author{
  Stephen McQuistin\\University of Glasgow
\and 
  Colin Perkins\\University of Glasgow
}
\date{\today}
\maketitle
%==================================================================================================
\begin{abstract}

This memo describes the requirements that generating parsers places on the Network
Representation Language. As currently defined, a number of the language features introduce
ambiguity into the parsing process: this document serves to surface these features.
\end{abstract}

%==================================================================================================
\section{Representable Types}

%--------------------------------------------------------------------------------------------------
\subsection{Bit String Types}

%--------------------------------------------------------------------------------------------------
\subsection{Arrays}

%--------------------------------------------------------------------------------------------------
\subsection{Structure Types}

%--------------------------------------------------------------------------------------------------
\subsection{Enumerated Types}

%--------------------------------------------------------------------------------------------------
\subsection{Derived Types}

%--------------------------------------------------------------------------------------------------
\subsection{Functions}

%--------------------------------------------------------------------------------------------------
\subsection{Protocols}

%==================================================================================================
\section{Acknowledgements}

This work was supported by the Engineering and Physical Sciences Research
Council (grant EP/R04144X/1).

%==================================================================================================
\bibliographystyle{abbrvurl}
\bibliography{ir}
%==================================================================================================
% The following information gets written into the PDF file information:
\ifpdf
  \pdfinfo{
    /Title        (The Network Representation Language: Parser Generator Requirements)
    /Author       (Stephen McQuistin and Colin Perkins)
    /Subject      (The Network Representation Language)
    /Keywords     (Parsing, Network Protocols, Packet Formats)
    /CreationDate (D:20180727162600Z)
    /ModDate      (D:20180727162600Z)
    /Creator      (LaTeX)
    /Producer     (pdfTeX)
  }
  % Suppress unnecessary metadata, to ensure the PDF generated by pdflatex is
  % identical each time it is built:
  \ifdefined\pdftrailerid
    % The \pdftrailerid and \pdfsuppressptexinfo macros were both introduced 
    % in pdfTeX 3.14159265-2.6-1.40.17. If one is present, the other will be.
    \pdftrailerid{}
    \pdfsuppressptexinfo=15
  \fi
\fi
%==================================================================================================
\end{document}
% vim: set ts=2 sw=2 tw=75 et ai:
