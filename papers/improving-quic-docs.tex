%==================================================================================================
% LaTeX paper template - use as a starting point for structuring a paper.
%
% Written by Colin Perkins (https://csperkins.org/)
% 2002-2017
%
% To the extent possible under law, the author(s) have dedicated all copyright and
% related and neighbouring rights to this software to the public domain worldwide.
% This software is distributed without any warranty.
%
% You should have received a copy of the CC0 Public Domain Dedication along with
% this software. If not, see <http://creativecommons.org/publicdomain/zero/1.0/>.
%
%==================================================================================================
% General advice on technical writing:
%  - George Gopen and Judith Swan, "The Science of Scientific Writing",
%    American Scientist, Nov/Dec 1990. 
%    http://www.americanscientist.org/issues/num2/the-science-of-scientific-writing/1
%  - Stephen Pinker, "The Sense of Style: The Thinking Person's Guide to
%    Writing in the 21st Century", Penguin, Sept 2014. ISBN 0525427929.
%
% The paper writing advice in the comments is derived from talks and articles
% by Simon Peyton-Jones, Jim Kurose, Henning Schulzrinne, and Jim Bednar: 
%  - http://research.microsoft.com/~simonpj/papers/giving-a-talk/giving-a-talk.htm
%  - http://research.microsoft.com/~simonpj/papers/giving-a-talk/writing-a-paper-slides.pdf
%  - http://gaia.cs.umass.edu/kurose/talks/top_10_tips_for_writing_a_paper.ppt
%  - http://www-net.cs.umass.edu/kurose/writing/intro-style.html
%  - http://www.cs.columbia.edu/~hgs/etc/writing-style.html
%  - http://homepages.inf.ed.ac.uk/jbednar/writingtips.html
%  - http://www.gabbay.org.uk/blog/paper-writing.html
%  - http://homes.cs.washington.edu/~mernst/advice/write-technical-paper.html
%  - https://doi.org/10.1371/journal.pcbi.1005619
%
%==================================================================================================
\documentclass[10pt,sigconf]{acmart}
% The \documentclass line will need to be updated to use the appropriate
% class for your chosen venue. 
% 
% For IEEE style use:
%   \documentclass[conference]{IEEEtran}
%
% For the ACM SIGCOMM style use:
%   \documentclass{sig-alternate}
% or
%   \documentclass{sig-alternate-10pt}
% 
% The IEEE and ACM templates are included in the lib/tex/inputs directory.
% The bin/latex-build.sh script is set to search this directory for class
% files before the versions included with your LaTeX distribution. 
% 
% The ACM templates (sig-alternate.cls and sig-alternate-10pt.cls) included
% have been edited to remove \usepackage{epsfig}, as the epsfig package has
% long been deprecated in favour of graphicx. They also include a number of
% other minor bug fixes. These modifications don't change the output format.

% The following packages are recommended, and should be available in most
% standard LaTeX installations (or from https://www.ctan.org/). Note that 
% the order in which these packages are loaded is significant.
\usepackage[l2tabu,orthodox]{nag}  % Warn about use of obsolete LaTeX packages/constructs
\usepackage[utf8x]{inputenc}       % This file is formatted in UTF-8
\usepackage[british]{babel}        % Use British hyphenation patterns
\usepackage{ifpdf}                 % Support conditional use of PDF-specific features
% Use the AMS mathematics library for improved mathematics support (see
% http://ctan.org/pkg/amsmath). When this package is loaded, use:
%   \begin{align}
%     ...
%   \end{align} 
% instead of:
%   \begin{eqnarray}
%     ...
%   \end{eqnarray}.  
% (see https://tug.org/pracjourn/2006-4/madsen/madsen.pdf)
\usepackage{amsmath}
\usepackage[all]{onlyamsmath}
% Use Times, Helvetica, and Courier fonts, rather than Computer Modern
\usepackage{upquote}   % Fix quotes in verbatim environments
\usepackage{graphicx}  % Enhanced graphics support
\usepackage{url}       % Add support for typesetting URLs using \url{...}:
% Add support for subfigures, as follows:
%   \begin{figure}
%     \centering
%     \subfloat[caption for 1st subfloat]{
%       \includegraphics{...}
%       \label{...}
%     }
%     \\
%     \subfloat[caption for 2nd subfloat]{
%       \includegraphics{...}
%       \label{...}
%     }
%     \caption{caption for entire figure}
%     \label{...}
%   \end{figure*}
% The subfig package obsoletes the older subfigure package, and is itself
% deprecated in favour of the subcaption package. However, as of April 2015
% subcaption doesn't work with ACM or IEEE style files (this is also the
% reason for the [caption=false] option).
\usepackage[caption=false]{subfig}
% Improve table formatting. This defines three new commands \toprule,
% \midrule, and \bottomrule that should be used instead of \hline in
% tabular environments to get well formatted tables.
\usepackage{booktabs}
% Add support for drawing network packet diagrams. For documentation, see
% http://ctan.org/tex-archive/macros/latex/contrib/bytefield
\usepackage{bytefield}
% Add support for typesetting program source code. For documentation, see
% http://ctan.org/tex-archive/macros/latex/contrib/listings
\usepackage{listings}

% The hyperref package is problematic. Known issues include:
%  - Papers typeset without page numbers gives warnings of the form:
%      "pdfTeX warning (ext4): destination with the same identifier 
%      (name{page.}) has been already used, duplicate ignored".
%    since hyperref tries to refer to the page number.
%  - The algorithmic package uses the same line-numbering scheme for each
%    algorithm, and can cause duplicate identifier warnings if you have
%    several algorithms with line numbers (this may have been fixed with 
%    recent versions of algorithmic...).
%  - If using the algorithm package with hyperref, you need to load packages
%    in the following order (see README in hyperref documentation):
%      \usepackage{float}
%      \usepackage{hyperref}
%      \usepackage{algorithm}
% For these reasons, hyperref is best to avoid for most papers, however if
% needed, uncomment the following two lines:
%  \usepackage{float}
%  \usepackage{hyperref}

% The algorithm package defines the algorithm environment. This is used in
% the same way as the figure and table environments, to include algorithms
% in a paper. The algpseudocode package provides the ability to typeset the
% algorithms: http://ctan.org/tex-archive/macros/latex/contrib/algorithmicx
\usepackage{algorithm}
\usepackage{algpseudocode}

\usepackage{color} % Add colour support:

% Define a simple \todo{...} macro:
\newcommand{\todo}[1]{\textbf{\textcolor{red}{To do: #1}}}

%==================================================================================================
\begin{document}
\title{Improving QUIC Protocol Documents}

\author{Stephen McQuistin}
\affiliation{%
  \institution{University of Glasgow}
  \city{Glasgow, UK} 
}
\email{sm@smcquistin.uk}

\author{Colin Perkins}
\affiliation{%
  \institution{University of Glasgow}
  \city{Glasgow, UK} 
}
\email{csp@csperkins.org}

\date{\today}

%==================================================================================================
\begin{abstract}

% Four sentences:
%  - State the problem
%  - Say why it's an interesting problem
%  - Say what your solution achieves
%  - Say what follows from your solution

\end{abstract}

\maketitle

%==================================================================================================
\section{Introduction}

% A good paper introduction is fairly formulaic. If you follow a simple set
% of rules, you can write a very good introduction. The following outline can
% be varied. For example, you can use two paragraphs instead of one, or you
% can place more emphasis on one aspect of the intro than another. But in all
% cases, all of the points below need to be covered in an introduction, and
% in most papers, you don't need to cover anything more in an introduction.
%
% Paragraph 1: Motivation. At a high level, what is the problem area you
% are working in and why is it important? It is important to set the larger
% context here. Why is the problem of interest and importance to the larger
% community?

ASCII packet header diagrams are the single-most widely adopted formalism in IETF
standards documents, allowing for the visualisation of packet formats. However, QUIC
\cite{draft-ietf-quic-transport-latest} includes a number of features -- for example,
packet protection and variable-length integer encoding -- that diminish the utility of
such diagrams. As a result, correct parser implementations are much more reliant upon
the careful interpretation of prose within the QUIC standards documents. This is a
situation that is likely to result in non-conformant implementations.

% Paragraph 2: What is the specific problem considered in this paper? This
% paragraph narrows down the topic area of the paper. In the first
% paragraph you have established general context and importance. Here you
% establish specific context and background.

Eliminating ambiguity in the description of protocols requires a formalism that goes
beyond ASCII packet header diagrams, allowing for increasingly complex protocol features
to be captured. However, such a formalism must not overly complicate the protocol
standardisation process: the poor adoption of previous packet format definition languages
serves as a lesson. Finally, any formalism should be flexible in how protocols are
described, and the types of artefacts that can be generated from those descriptions.

% Paragraph 3: "In this paper, we show that...". This is the key paragraph
% in the introduction - you summarize, in one paragraph, what are the main
% contributions of your paper, given the context established in paragraphs 
% 1 and 2. What's the general approach taken? Why are the specific results
% significant? The story is not what you did, but rather:
%  - what you show, new ideas, new insights
%  - why interesting, important?
% State your contributions: these drive the entire paper.  Contributions
% should be refutable claims, not vague generic statements.

In this paper, we ..

% Paragraph 4: What are the differences between your work, and what others
% have done? Keep this at a high level, as you can refer to future sections
% where specific details and differences will be given, but it is important
% for the reader to know what is new about this work compared to other work
% in the area.



% Paragraph 5: "We structure the remainder of this paper as follows." Give
% the reader a road-map for the rest of the paper. Try to avoid redundant
% phrasing, "In Section 2, In section 3, ..., In Section 4, ... ", etc.

We structure the remainder of this paper as follows.

%==================================================================================================
% Concentrate single-mindedly on a narrative that:
%  - Describes the problem, and why it's interesting
%  - Describes your idea
%  - Defends your idea, showing how it solves the problem, and filling out
%    the details
% On the way, cite relevant work in passing, but defer discussion to the
% end.
%
% Introduce the problem, and your idea, using examples, and only then
% present the general case. Explain the idea as if your were speaking to
% someone using a whiteboard. Conveying the intuition is primary; details
% follow. Write in a top down manner: state broad themes and ideas first,
% then go into details.
%
% The introduction makes claims. The body of the paper provides evidence
% to support each claim. Check each claim in the introduction, identify
% the evidence, and forward-reference it from the claim. 
%==================================================================================================

\section{Motivation}

%==================================================================================================
\section{SYSTEMNAME}

\subsection{Contexts}

\subsection{Interfaces}

%==================================================================================================
\section{Example: QUIC's Short Header}

% The details

% Describe results carefully:
%  - clearly state assumptions
%  - give enough information to allow the reader to recreate the results
%  - ensure results are representative; statistically meaningful, etc.
%  - don't overstate results
%  - equally, don't understate them: consider the broader implications

%==================================================================================================
\section{Related Work}

% This should come near the end, and focussing on discussing how your work
% relates to that of others. Any relevant related work should have been
% cited already, so this is not a list of related work, it's a discussion
% of how that work relates.
%
% Why not put related work after the introduction? 1) because describing
% alternative approaches gets between the reader and your idea; and 2)
% because the reader knows nothing about the problem yet, so your
% (carefully trimmed) description of various technical trade-offs is
% absolutely incomprehensible.
% 
% When writing the related work:
%  - Give credit to others where it's due; this doesn't diminish the
%    credit you get from your paper. 
%  - Acknowledge weaknesses in your approach.
%  - Ensure related work is accurate and up-to-date


%==================================================================================================
\section{Conclusions}



%==================================================================================================
\section{Acknowledgements}

This work is funded by the UK Engineering and Physical Sciences Research Council, under
grant EP/R04144X/1.

%==================================================================================================
% Set the bibliography style. 
\bibliographystyle{abbrvurl}
% The abbrvurl style comes from https://www.ctan.org/pkg/urlbst, and is
% included in most standard TeX distributions. It adds support for DOIs
% to the standard abbrv style.
%
% When using the IEEE style, replace the above line with:
%   \bibliographystyle{IEEEtran} 
% The IEEEtran.bst file included here has been modified to support DOIs.

% Load the bibliography file for this paper:
\bibliography{improving-quic-docs}

%==================================================================================================
% The following information gets written into the PDF file information:
\ifpdf
  \pdfinfo{
    /Title        (...)
    /Author       (...)
    /Subject      (...)
    /Keywords     (..., ..., ...)
    /CreationDate (D:20150827110616Z)
    /ModDate      (D:20150827110616Z)
    /Creator      (LaTeX)
    /Producer     (pdfTeX)
  }
  % Suppress unnecessary metadata, to ensure the PDF generated by pdflatex is
  % identical each time it is built:
  \ifdefined\pdftrailerid
    % The \pdftrailerid and \pdfsuppressptexinfo macros were both introduced 
    % in pdfTeX 3.14159265-2.6-1.40.17. If one is present, the other will be.
    \pdftrailerid{}
    \pdfsuppressptexinfo=15
  \fi
\fi
%==================================================================================================
\end{document}
% vim: set ts=2 sw=2 tw=75 et ai: